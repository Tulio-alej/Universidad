\documentclass{article}
\usepackage[utf8]{inputenc}
\usepackage[spanish]{babel}
\usepackage{booktabs}
\usepackage{listings}
\usepackage{xcolor}

\definecolor{codegreen}{rgb}{0,0.6,0}
\definecolor{codegray}{rgb}{0.5,0.5,0.5}
\definecolor{codepurple}{rgb}{0.58,0,0.82}
\definecolor{backcolour}{rgb}{0.95,0.95,0.92}

\lstdefinestyle{mips}{
    backgroundcolor=\color{backcolour},   
    commentstyle=\color{codegreen},
    keywordstyle=\color{magenta},
    numberstyle=\tiny\color{codegray},
    stringstyle=\color{codepurple},
    basicstyle=\ttfamily\footnotesize,
    breakatwhitespace=false,         
    breaklines=true,                 
    captionpos=b,                    
    keepspaces=true,                 
    numbers=left,                    
    numbersep=5pt,                  
    showspaces=false,                
    showstringspaces=false,
    showtabs=false,                  
    tabsize=2,
    frame=single
}

\lstset{style=mips}

\title{Guía MIPS32: Registros, Instrucciones}

\begin{document}

\maketitle

\section{Registros MIPS}

\begin{table}[h]
\centering

\begin{tabular}{lllp{6cm}}
\toprule
\textbf{Nombre} & \textbf{Número} & \textbf{¿se mantiene?} & \textbf{Explicación} \\
\midrule
\$zero & 0 & Sí (siempre 0) & Siempre contiene el valor 0. No se puede modificar. \\
\$at & 1 & No & Registro temporal reservado para el ensamblador (pseudoinstrucciones) \\
\$v0-\$v1 & 2-3 & No & Valores de retorno de funciones (return values) \\
\$a0-\$a3 & 4-7 & No & Argumentos para llamadas a funciones (arguments) \\
\$t0-\$t7 & 8-15 & No & Registros temporales (no preservados entre llamadas) \\
\$s0-\$s7 & 16-23 & Sí & Registros guardados (preservados entre llamadas) \\
\$k0-\$k1 & 26-27 & No & Reservados para el núcleo del sistema operativo \\
\$gp & 28 & Sí & Puntero global (global pointer) \\
\$sp & 29 & Sí & Puntero de pila (stack pointer) \\
\$fp & 30 & Sí & Puntero de marco (frame pointer) \\
\$ra & 31 & Sí & Dirección de retorno (return address) \\
\bottomrule
\end{tabular}
\end{table}

\begin{itemize}
\item \textbf{Número}: Cada registro tiene un identificador numérico (0-31) que se usa en la codificación binaria de las instrucciones. Por ejemplo, \$t0 es el registro 8.

\item \textbf{¿Se mantiene?}: Indica si el registro debe preservar su valor después de una llamada a función (convención de llamadas):
\begin{itemize}
\item \textbf{Sí}: La función llamada debe dejar el registro con el mismo valor que tenía al entrar (usualmente se guarda en la pila).
\item \textbf{No}: La función llamada puede modificar el registro sin restricciones.
\end{itemize}
\end{itemize}

\section{Instrucciones Aritméticas}

\subsection{Aritmética Básica}
\begin{lstlisting}[language={[mips]Assembler}]
add $t0, $t1, $t2    # $t0 = $t1 + $t2 (suma con signo)
addu $t0, $t1, $t2   # $t0 = $t1 + $t2 (suma sin signo)
sub $t0, $t1, $t2    # $t0 = $t1 - $t2 (resta con signo)
subu $t0, $t1, $t2   # $t0 = $t1 - $t2 (resta sin signo)
addi $t0, $t1, 5     # $t0 = $t1 + 5 (suma inmediata)
\end{lstlisting}

\subsection{Multiplicación y División}
\begin{lstlisting}[language={[mips]Assembler}]
mult $t0, $t1        # Hi/Lo = $t0 * $t1 (producto de 32 bits)
div $t0, $t1         # Lo = $t0/$t1, Hi = $t0 % $t1 (division con signo)
mflo $t2             # $t2 = Lo (copia registro bajo)
mfhi $t3             # $t3 = Hi (copia registro alto)
\end{lstlisting}

\section{Instrucciones de Comparación y Saltos}

\subsection{Comparaciones}
\begin{lstlisting}[language={[mips]Assembler}]
slt $t0, $t1, $t2    # $t0 = 1 si $t1 < $t2 (con signo), 0 en otro caso
sltu $t0, $t1, $t2   # $t0 = 1 si $t1 < $t2 (sin signo), 0 en otro caso
slti $t0, $t1, 10    # $t0 = 1 si $t1 < 10 (inmediato con signo)
\end{lstlisting}

\subsection{Saltos Condicionales}
\begin{lstlisting}[language={[mips]Assembler}]
beq $t0, $t1, etiqueta  # Salta si $t0 == $t1
bne $t0, $t1, etiqueta  # Salta si $t0 != $t1
beqz $t0, etiqueta      # Salta si $t0 == 0 (pseudoinstruccion)
bgt $t0, $t1, etiqueta  # Salta si $t0 > $t1 (pseudoinstruccion)
blt $t0, $t1, etiqueta  # Salta si $t0 < $t1 (pseudoinstruccion)
bge $t0, $t1, etiqueta  # Salta si $t0 >= $t1 (pseudoinstruccion)
ble $t0, $t1, etiqueta  # Salta si $t0 <= $t1 (pseudoinstruccion)
\end{lstlisting}

\subsection{Saltos Incondicionales}
\begin{lstlisting}[language={[mips]Assembler}]
j etiqueta           # Salto incondicional
jal etiqueta         # Salto y enlace (para llamadas a funcion)
jr $ra               # Salto a dirección en $ra (retorno de funcion)
\end{lstlisting}

\section{Manejo de Memoria}

\subsection{Concepto de Palabra}
En MIPS, una \textbf{palabra} (word) son 4 bytes (32 bits). Las direcciones de memoria deben estar alineadas a palabras (múltiplos de 4).

\subsection{Carga y Almacenamiento}
\begin{lstlisting}[language={[mips]Assembler}]
lw $t0, 0($sp)      # Carga palabra de ($sp + 0) en $t0
sw $t0, 4($sp)      # Almacena $t0 en ($sp + 4)
lb $t0, 0($sp)      # Carga byte (sign-extend) en $t0
lbu $t0, 0($sp)     # Carga byte (zero-extend) en $t0
sb $t0, 0($sp)      # Almacena byte (los 8 bits bajos de $t0)
\end{lstlisting}

\section{Entrada/Salida Básica}

\subsection{Lectura y Escritura de Enteros}
\begin{lstlisting}[language={[mips]Assembler}]
# Ejemplo: Leer un entero y mostrarlo
.data
    mensaje: .asciiz "Ingrese un numero: "
    resultado: .asciiz "El numero ingresado es: "

.text
main:
    # Mostrar mensaje
    li $v0, 4           # Codigo para imprimir string
    la $a0, mensaje      # Carga dirección del mensaje
    syscall

    # Leer entero
    li $v0, 5           # Codigo para leer entero
    syscall
    move $t0, $v0       # Guardar el valor leído en $t0

    # Mostrar resultado
    li $v0, 4
    la $a0, resultado
    syscall

    li $v0, 1           # Codigo para imprimir entero
    move $a0, $t0       # Cargar el valor a imprimir
    syscall

    # Terminar programa
    li $v0, 10
    syscall
\end{lstlisting}

\subsection{Manejo de Caracteres}
\begin{lstlisting}[language={[mips]Assembler}]
.data
    char: .byte 'A'     # Caracter ASCII
    buffer: .space 20   # Buffer para cadena

.text
main:
    # Leer un carácter
    li $v0, 12          # Codigo para leer caracter
    syscall
    sb $v0, char        # Almacenar caracter leido

    # Mostrar carácter
    li $v0, 11          # Codigo para imprimir caracter
    lb $a0, char        # Cargar caracter
    syscall

    # Leer cadena
    li $v0, 8           # Codigo para leer cadena
    la $a0, buffer      # Direccion del buffer
    li $a1, 20          # Longitud maxima
    syscall

    # Mostrar cadena
    li $v0, 4
    la $a0, buffer
    syscall
\end{lstlisting}

\subsection{Números Decimales (Punto Flotante)}
\begin{lstlisting}[language={[mips]Assembler}]
.data
    flotante: .float 3.1416
    doble: .double 2.71828
    mensaje: .asciiz "El valor es: "

.text
main:
    # Leer float
    li $v0, 6           # Codigo para leer float
    syscall             # Resultado en $f0

    # Mostrar float
    li $v0, 4
    la $a0, mensaje
    syscall

    li $v0, 2           # Codigo para imprimir float
    mov.s $f12, $f0     # Cargar valor a imprimir
    syscall

    # Leer double
    li $v0, 7           # Codigo para leer double
    syscall             # Resultado en $f0-$f1

    # Mostrar double
    li $v0, 4
    la $a0, mensaje
    syscall

    li $v0, 3           # Codigo para imprimir double
    mov.d $f12, $f0     # Cargar valor a imprimir
    syscall
\end{lstlisting}

\end{document}